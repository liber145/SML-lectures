\documentclass[11pt]{article}
\usepackage{graphicx} % more modern
%\usepackage{times}
\usepackage{helvet}
\usepackage{courier}
\usepackage{epsf}
\usepackage{amsmath,amssymb,amsfonts,verbatim}
\usepackage{amsthm}
\usepackage{subfigure}
\usepackage{amsfonts}
\usepackage{latexsym}
\usepackage{algpseudocode}
\usepackage{algorithm}
%\usepackage{algorithmic}
\usepackage{multirow}
\usepackage{xcolor}

\def\A{{\bf A}}
\def\a{{\bf a}}
\def\B{{\bf B}}
\def\b{{\bf b}}
\def\C{{\bf C}}
\def\c{{\bf c}}
\def\D{{\bf D}}
\def\d{{\bf d}}
\def\E{{\bf E}}
\def\e{{\bf e}}
\def\F{{\bf F}}
\def\f{{\bf f}}
\def\G{{\bf G}}
\def\g{{\bf g}}
\def\k{{\bf k}}
\def\K{{\bf K}}
\def\H{{\bf H}}
\def\I{{\bf I}}
\def\L{{\bf L}}
\def\M{{\bf M}}
\def\m{{\bf m}}
\def\n{{\bf n}}
\def\N{{\bf N}}
\def\BP{{\bf P}}
\def\R{{\bf R}}
\def\BS{{\bf S}}
\def\s{{\bf s}}
\def\t{{\bf t}}
\def\T{{\bf T}}
\def\U{{\bf U}}
\def\u{{\bf u}}
\def\V{{\bf V}}
\def\v{{\bf v}}
\def\W{{\bf W}}
\def\w{{\bf w}}
\def\X{{\bf X}}
\def\Y{{\bf Y}}
\def\Q{{\bf Q}}
\def\x{{\bf x}}
\def\y{{\bf y}}
\def\Z{{\bf Z}}
\def\z{{\bf z}}
\def\0{{\bf 0}}
\def\1{{\bf 1}}


\def\hx{\hat{\bf x}}
\def\tx{\tilde{\bf x}}
\def\ty{\tilde{\bf y}}
\def\tz{\tilde{\bf z}}
\def\hd{\hat{d}}
\def\HD{\hat{\bf D}}

\def\MF{{\mathcal F}}
\def\MG{{\mathcal G}}
\def\MI{{\mathcal I}}
\def\MN{{\mathcal N}}
\def\MO{{\mathcal O}}
\def\MT{{\mathcal T}}
\def\MX{{\mathcal X}}
\def\SW{{\mathcal {SW}}}
\def\MW{{\mathcal W}}
\def\MY{{\mathcal Y}}
\def\BR{{\mathbb R}}
\def\BE{{\mathbb E}}
\def\BP{{\mathbb P}}

\def\bet{\mbox{\boldmath$\beta$\unboldmath}}
\def\epsi{\mbox{\boldmath$\epsilon$}}

\def\etal{{\em et al.\/}\,}
\def\tr{\mathrm{tr}}
\def\rk{\mathrm{rk}}
\def\diag{\mathrm{diag}}
\def\dg{\mathrm{dg}}
\def\argmax{\mathop{\rm argmax}}
\def\argmin{\mathop{\rm argmin}}
\def\vecd{\mathrm{vec}}

\def\ph{\mbox{\boldmath$\phi$\unboldmath}}
\def\vp{\mbox{\boldmath$\varphi$\unboldmath}}
\def\pii{\mbox{\boldmath$\pi$\unboldmath}}
\def\Ph{\mbox{\boldmath$\Phi$\unboldmath}}
\def\pss{\mbox{\boldmath$\psi$\unboldmath}}
\def\Ps{\mbox{\boldmath$\Psi$\unboldmath}}
\def\muu{\mbox{\boldmath$\mu$\unboldmath}}
\def\Si{\mbox{\boldmath$\Sigma$\unboldmath}}
\def\lam{\mbox{\boldmath$\lambda$\unboldmath}}
\def\Lam{\mbox{\boldmath$\Lambda$\unboldmath}}
\def\Gam{\mbox{\boldmath$\Gamma$\unboldmath}}
\def\Oma{\mbox{\boldmath$\Omega$\unboldmath}}
\def\De{\mbox{\boldmath$\Delta$\unboldmath}}
\def\de{\mbox{\boldmath$\delta$\unboldmath}}
\def\Tha{\mbox{\boldmath$\Theta$\unboldmath}}
\def\tha{\mbox{\boldmath$\theta$\unboldmath}}

\newtheorem{theorem}{Theorem}[section]
\newtheorem{lemma}{Lemma}[section]
%\newtheorem{proof}{Proof}[section]
\newtheorem{definition}{Definition}[section]
\newtheorem{proposition}{Proposition}[section]
\newtheorem{corollary}{Corollary}[section]
\newtheorem{example}{Example}[section]


\def\probin{\mbox{\rotatebox[origin=c]{90}{$\vDash$}}}

\def\calA{{\cal A}}



%this is a comment

%use this as a template only... you may not need the subsections,
%or lists however they are placed in the document to show you how
%do it if needed.


%THINGS TO REMEMBER
%to compile a latex document - latex filename.tex
%to view the document        - xdvi filename.dvi
%to create a ps document     - dvips filename.dvi
%to create a pdf document    - dvipdf filename.dvi
%{\bf TEXT}                  - bold font TEXT
%{\it TEXT}                  - italic TEXT
%$ ... $                     - places ... in math mode on same line
%$$ ... $$                   - places ... in math mode on new line
%more info at www.cs.wm.edu/~mliskov/cs423_fall04/tex.html


\setlength{\oddsidemargin}{.25in}
\setlength{\evensidemargin}{.25in}
\setlength{\textwidth}{6in}
\setlength{\topmargin}{-0.4in}
\setlength{\textheight}{8.5in}


%%%%%%%%%%%%%%%%%%%%%%%%%%%%%%%%%%%%%%%%%%%%%%%%%%%%%%%%%%%%%%%%%%%%%%%%%%%%%%%%%%%
\newcommand{\notes}[5]{
	\renewcommand{\thepage}{#1 - \arabic{page}}
	\noindent
	\begin{center}
	\framebox{
		\vbox{
		\hbox to 5.78in { { \bf Statistic Machine Learning}
		\hfill #2}
		\vspace{4mm}
		\hbox to 5.78in { {\Large \hfill #5 \hfill} }
		\vspace{2mm}
		\hbox to 5.78in { {\it #3 \hfill #4} }
		}
	}
	\end{center}
	\vspace*{4mm}
}

\newcommand{\ho}[1]{\notes{#1}{Probability Inequalities(2)}{Professor: Zhihua Zhang}{Scribe:}{Lecture Notes #1: Probability Inequalities(2)}}
%%%%%%%%%%%%%%%%%%%%%%%%%%%%%%%%%%%%%%%%%%%%%%%%%%%%%%%%%%%%%%%%%%%%%%%%%%%%%%%%%%
\setcounter{section}{9}
%begins a LaTeX document

\begin{document}

\ho{10}

\section{Probability Inequalities(2)}

\subsection{Hoeffding's Inequality}


If $X_1$, ... , $X_n$ are indenpendent random variables with a finite mean value such that for some non-empty interval $I$, $\BE e^{\lambda X_i}$ is finite. Then we define 
\[ S = \sum_{i=1}^n (X_i - \BE X_i)\].
Assume $X_i$ takes its values in a bounded interval $[a_i,b_i]$. Then \[
\BP(S \geq t) \leq \exp\left( -\frac{2t^2}{\sum_{i=1}^n (b_i - a_i)^2} \right) \]
for all $t > 0$.


\begin{lemma} (\textbf{Hoeffding's Lemma})
Let $Y$ be a random variable with $\BE Y = 0$, taking values in a bounded interval $[a, b]$ and let $\psi_Y(\lambda) = \log \BE e^{\lambda Y}$. Then $\psi''_Y(\lambda) \leq \frac{(b-a)^2}{4}$ and $\psi_Y(\lambda) \leq \frac{\lambda^2}{2}\frac{(b-a)^2}{4}$.
\end{lemma}

We first show that if Hoeffding's lemma is true, then Hoeffding's inequality is also true.

\begin{proof}(Hoeffding's Lemma $\rightarrow$ Hoeffding's Inequality)

Let $Y_i = X_i - \BE X_i$. So $\BE Y_i = 0$ ,$S = \sum_{i=1}^{n} Y_i$ and $Y_i \in [a_i - \BE Y_i, b_i - \BE Y_i]$.

\[\begin{split}
\Pr (S \geq t) &\leq \frac{\BE e^{\lambda S}}{e^{\lambda t}} \\
&=\exp(- \lambda t + \log\BE e^{\lambda S}) \\ 
& (Y_i's \text{ are independent}) \\
&= \exp(-\lambda t + \log \Pi_{i=1}^n \BE e^{\lambda Y_i}) \\
& = \exp(-\lambda t + \sum_{i = 1}^n \log \BE e^{\lambda Y_i}) \\
& \leq \exp(-\lambda t + \frac{\lambda^2}{2} \sum_{i=1}^n\frac{(b_i - a_i)^2}{4}) \\
&\leq \exp(- \frac{2t^2}{\sum_{i=1}^n} (b_i - a_i)^2)
\end{split}\]
\end{proof}

Next, we will give two proofs for Hoeffding's Lemma.

\begin{proof}
Let $g(y) = \exp(-\psi_Y(\lambda)) e^{\lambda y} f(y)$, then we have 
\[
\int g(y) dy = \frac{\int e^{\lambda y} f(y) dy}{\exp(\psi_Y(\lambda))} = 1
\]

Because $Y \in [a, b]$, we can get $|Y - \frac{a+b}{2}| \leq \frac{b - a}{a}$. So $Var(Y) = \BE ((Y - \frac{a+b}{2})^2) \leq \frac{(b-a)^2}{4}$.

And we have
\[\begin{split} 
\BE ( Y - \BE_g Y)^2 & = 
\int (y^2 - 2y\BE_gY + \BE_g^2(Y)) g(y) dy \\
&= \exp(-\psi_Y(\lambda )) \int y^2 e^{\lambda y} f(y) dy - 
2\exp(-2\psi_Y(\lambda ))\int y e^{\lambda y}f(y) \int y e^{\lambda y} f(y)dy dy + \BE_g^2(Y) \\
&= \exp(-\psi_Y(\lambda )) \int y^2 e^{\lambda y} f(y) dy -
2 [\exp(-\psi_Y(\lambda))]^2 \left(\int ye^{\lambda y}f(y) dy\right)^2 + \\
& [\exp(-\psi_Y(\lambda))]^2 \left(\int ye^{\lambda y}f(y) dy\right)^2 \\
&= \exp(-\psi_Y(\lambda )) \int y^2 e^{\lambda y} f(y) dy -
[\exp(-\psi_Y(\lambda))]^2 \left(\int ye^{\lambda y}f(y) dy\right)^2\\
& = \exp(-\psi_Y(\lambda)) \BE_f (Y^2 e^{\lambda Y}) - \exp(-2\psi_Y(\lambda))(\BE_f(Ye^{\lambda Y}))^2 \\
&\leq \frac{(b-a)^2}{4}
\end{split}\]

By Taylor's expansion, we have
\[
\psi_Y(\lambda ) = \psi_Y(0) + \lambda\psi'_Y(0) +  \frac{\lambda^2}{2} \psi''_Y(\theta)
\]
where $\theta \in (0, \lambda)$.
And $\psi_Y(0) = \psi'_Y(0) = 0$ and 
$\psi''_Y(\lambda) = \frac{\BE_f (Y^2 e^{\lambda Y})}{\BE_f e^{\lambda Y}} - \frac{[\BE_f(Ye^{\lambda Y})]^2}{\BE_f e^{2\lambda Y}} \leq \frac{(b-a)^2}{4}$.
So,
\[\begin{split}
\psi_Y(\lambda) &= \frac{\lambda^2}{2}\psi''_Y(\theta) \\
&\leq \frac{\lambda^2}{2}\psi''_Y(\lambda) \\
&\leq \frac{\lambda^2}{2} \frac{(b-a)^2}{4}
\end{split} \]
\end{proof}

Let $X$ be any real-valued random variable with expected value $\BE X = 0$ and such that $a \leq X \leq b$ almost surely. Then, for all $\lambda \in \BR$,\[ \BE[e^{\lambda X}] \leq \exp (\frac{\lambda^2 (b-a)^2}{8}) \]

\begin{proof}
 Since $e^{\lambda x}$ is a convex function, we have 
 \[ e^{\lambda x} \leq \frac{b - x}{b - a} e^{\lambda a} + \frac{x - a}{b - a} e^{\lambda b}, \forall a \leq x \leq b \]
 So, 
 \[ \BE[e^{\lambda X}] \leq \frac{b - \BE X}{b - a} e^{\lambda a} + \frac{\BE X - a}{b - a} e^{\lambda b}. \]
 Let $\alpha = \frac{-a}{b - a} \in [0, 1]$, $u = \lambda(b -a)$, and $L(u) = -\alpha u + \ln (1 - \alpha + \alpha e^u)$
 
 Then $\frac{b - \BE X}{b - a} e^{\lambda a} + \frac{\BE X - a}{b - a} e^{\lambda b} = e^{L(u)}$ since $\BE X = 0$. 
 Taking derivative of $L(u)$,
 \[ L(0) = L'(0) = 0 \text{ and } L''(h) \leq \frac{1}{4} \]
 By Taylor's expansion, 
 \[ L(u) \leq \frac{1}{8}u^2 = \frac{1}{8}\lambda^2(b-a)^2 \]
 Hence, $\BE[e^{\lambda X}] \leq e^{\frac{1}{8} \lambda^2 (b-a)^2}$
\end{proof}


\subsection{Bennett's Inquality}

Let $X_1$, ..., $X_n$ be independent random variables with finite variance such that $X_i \leq b$ for some $b > 0$ almost surely for $i \leq n$. And let $v = \sum_{i=1}^n \BE(X_i^2)$. Assume $\psi(u) = e^u - u - 1$. For $u \in \BR$, and $\forall \lambda > 0$.
\[
\psi_S(\lambda) = \log \BE e^{\lambda S} \leq n \log (1 + \frac{v}{nb^2} \psi(b\lambda)) \leq \frac{v}{b^2} \psi(b\lambda).
\]
And for any $t > 0$,
\[
\BP (S \geq t) \leq \exp(-\frac{v}{b^2} h(\frac{bt}{v}))
\]
where $h(u) = (1 + u)\log(1+u) - u$ for $u > 0$.


\begin{theorem}\textbf{(Bernstein's Inequality)}
Let $h(x) = (1+x)\log(1+x) - x$ for $x \geq 0$. And $h(x) \geq \frac{x^2}{2(1 + \frac{x}{3})} = g(x)$.
So,
\[
\BP(S \geq t) \leq \exp\left( - \frac{t^2}{2 (v+\frac{bt}{3})}\right)
\]

\end{theorem}

\begin{proof}
$h'(x) = \log(1+x)$ and $h''(x) = \frac{1}{x+1}$

$g'(x) = \frac{x}{1 + \frac{x}{3}} - \frac{x^2}{6(1+\frac{x}{3})^2}$ and $g''(x) = \frac{27}{(x+3)^3}$

It's easy to see that $h^{(n)}(0) \geq g^{(n)}(0)$.
\end{proof}

\begin{lemma}\textbf{(Johnson-Lindenstrauss Lemma)}
Given $0 < \epsilon < 1$, a set $X$ of $m$ points in $\BR^N$, and a number $n > 8 \ln(m) / \epsilon^2$, there is a linear map $f: \BR^N \to \BR^n$ such that 
\[
(1 - \epsilon) ||u - v||^2 \leq ||f(u) - f(v)||^2 \leq (1 + \epsilon) \leq (1+\epsilon) ||u - v||^2
\]
for all $u, v \in X$.
\end{lemma}

\begin{definition}\textbf{(Sub-Gaussian Random Variables)}
A real-valued random variable $X$ is said to be subgaussian if it has the property that there is some $b > 0$ such that for every $\lambda \in \BR$ one has \[
\BE e^{\lambda X} \leq e^{\lambda^2 t^2/ 2}
\]
We denote that $X \in G(t^2)$.
\end{definition}

\begin{proposition}
If $X\in G(t^2)$, then 
\begin{enumerate}
\item $\BE X = 0$
\item $Var(X) \leq t^2$
\end{enumerate}
\end{proposition}
\begin{proof}
Using Taylor's expansion for the exponential function
\[
\BE \sum_{n = 0}^\infty \frac{\lambda^n x^n}{n!} = \BE e^{\lambda X}
\] and Lebesgue's Dominated Convergence Theorem, for any $\lambda \in \BR$,
\[
\sum_{n=0}^\infty \frac{\lambda^n}{n!} \BE(X^n) = \BE e^{\lambda X} \leq e^{\lambda^2 t^2 /2 } = \sum_{n = 0}^\infty \frac{\lambda^{2n}t^{2n}}{2^n n!}
\]
Thus 
\[
\lambda \BE X + \frac{\lambda^2}{2}\BE X^2 \leq \frac{\lambda^2 t^2}{2} + o(\lambda^2)
\text{\quad\quad as $\lambda \to 0$}
\]
Dividing through by $\lambda > 0$ ans letting $\lambda \to 0$ we get $\BE (X) \leq 0$. Dividing through by $\lambda < 0$ and letting $\lambda \to 0$ we get $\BE (X) \geq 0$. Thus $\BE(X) = 0$. Now that this is established, we divided through by $\lambda^2$ and let $\lambda \to 0$, thus getting $Var(X) \leq t^2$.
\end{proof}

\begin{example}
Let $X$ be a random variable with the Rademacher distribution, meaning that the law of $X$ if $\BP X = \frac{1}{2} \delta_{-1} + \frac{1}{2} \delta_1$[ here $\delta_t$ is the point mass at $x$]. Then for any $t \in \BR$, \[BE e^{tX} = \frac{1}{2} e^{-t} + \frac{1}{2} e^t = cosh t \leq e^{t^2 / 2}\]. So $X \in G(1)$.
\end{example}


\end{document}





