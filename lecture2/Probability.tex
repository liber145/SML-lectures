\documentclass[11pt]{article}
\usepackage{graphicx} % more modern
%\usepackage{times}
\usepackage{helvet}
\usepackage{courier}
\usepackage{epsf}
\usepackage{amsmath,amssymb,amsfonts,verbatim}
\usepackage{subfigure}
\usepackage{amsfonts}
\usepackage{amsmath}
\usepackage{latexsym}
\usepackage{algpseudocode}
\usepackage{algorithm}
%\usepackage{algorithmic}
\usepackage{multirow}
\usepackage{xcolor}

\def\A{{\bf A}}
\def\a{{\bf a}}
\def\B{{\bf B}}
\def\b{{\bf b}}
\def\C{{\bf C}}
\def\c{{\bf c}}
\def\D{{\bf D}}
\def\d{{\bf d}}
\def\E{{\bf E}}
\def\e{{\bf e}}
\def\F{{\bf F}}
\def\f{{\bf f}}
\def\G{{\bf G}}
\def\g{{\bf g}}
\def\k{{\bf k}}
\def\K{{\bf K}}
\def\H{{\bf H}}
\def\I{{\bf I}}
\def\L{{\bf L}}
\def\M{{\bf M}}
\def\m{{\bf m}}
\def\n{{\bf n}}
\def\N{{\bf N}}
\def\BP{{\bf P}}
\def\R{{\bf R}}
\def\BS{{\bf S}}
\def\s{{\bf s}}
\def\t{{\bf t}}
\def\T{{\bf T}}
\def\U{{\bf U}}
\def\u{{\bf u}}
\def\V{{\bf V}}
\def\v{{\bf v}}
\def\W{{\bf W}}
\def\w{{\bf w}}
\def\X{{\bf X}}
\def\Y{{\bf Y}}
\def\Q{{\bf Q}}
\def\x{{\bf x}}
\def\y{{\bf y}}
\def\Z{{\bf Z}}
\def\z{{\bf z}}
\def\0{{\bf 0}}
\def\1{{\bf 1}}


\def\hx{\hat{\bf x}}
\def\tx{\tilde{\bf x}}
\def\ty{\tilde{\bf y}}
\def\tz{\tilde{\bf z}}
\def\hd{\hat{d}}
\def\HD{\hat{\bf D}}

\def\MA{{\mathcal A}}
\def\MF{{\mathcal F}}
\def\MG{{\mathcal G}}
\def\MI{{\mathcal I}}
\def\MN{{\mathcal N}}
\def\MO{{\mathcal O}}
\def\MT{{\mathcal T}}
\def\MX{{\mathcal X}}
\def\SW{{\mathcal {SW}}}
\def\MW{{\mathcal W}}
\def\MY{{\mathcal Y}}
\def\BR{{\mathbb R}}
\def\BP{{\mathbb P}}

\def\bet{\mbox{\boldmath$\beta$\unboldmath}}
\def\epsi{\mbox{\boldmath$\epsilon$}}

\def\etal{{\em et al.\/}\,}
\def\tr{\mathrm{tr}}
\def\rk{\mathrm{rk}}
\def\diag{\mathrm{diag}}
\def\dg{\mathrm{dg}}
\def\argmax{\mathop{\rm argmax}}
\def\argmin{\mathop{\rm argmin}}
\def\vecd{\mathrm{vec}}

\def\ph{\mbox{\boldmath$\phi$\unboldmath}}
\def\vp{\mbox{\boldmath$\varphi$\unboldmath}}
\def\pii{\mbox{\boldmath$\pi$\unboldmath}}
\def\Ph{\mbox{\boldmath$\Phi$\unboldmath}}
\def\pss{\mbox{\boldmath$\psi$\unboldmath}}
\def\Ps{\mbox{\boldmath$\Psi$\unboldmath}}
\def\muu{\mbox{\boldmath$\mu$\unboldmath}}
\def\Si{\mbox{\boldmath$\Sigma$\unboldmath}}
\def\lam{\mbox{\boldmath$\lambda$\unboldmath}}
\def\Lam{\mbox{\boldmath$\Lambda$\unboldmath}}
\def\Gam{\mbox{\boldmath$\Gamma$\unboldmath}}
\def\Oma{\mbox{\boldmath$\Omega$\unboldmath}}
\def\De{\mbox{\boldmath$\Delta$\unboldmath}}
\def\de{\mbox{\boldmath$\delta$\unboldmath}}
\def\Tha{\mbox{\boldmath$\Theta$\unboldmath}}
\def\tha{\mbox{\boldmath$\theta$\unboldmath}}

\newtheorem{theorem}{Theorem}[section]
\newtheorem{lemma}{Lemma}[section]
\newtheorem{definition}{Definition}[section]
\newtheorem{proposition}{Proposition}[section]
\newtheorem{corollary}{Corollary}[section]
\newtheorem{example}{Example}[section]


\def\probin{\mbox{\rotatebox[origin=c]{90}{$\vDash$}}}

\def\calA{{\cal A}}



%this is a comment

%use this as a template only... you may not need the subsections,
%or lists however they are placed in the document to show you how
%do it if needed.


%THINGS TO REMEMBER
%to compile a latex document - latex filename.tex
%to view the document        - xdvi filename.dvi
%to create a ps document     - dvips filename.dvi
%to create a pdf document    - dvipdf filename.dvi
%{\bf TEXT}                  - bold font TEXT
%{\it TEXT}                  - italic TEXT
%$ ... $                     - places ... in math mode on same line
%$$ ... $$                   - places ... in math mode on new line
%more info at www.cs.wm.edu/~mliskov/cs423_fall04/tex.html


\setlength{\oddsidemargin}{.25in}
\setlength{\evensidemargin}{.25in}
\setlength{\textwidth}{6in}
\setlength{\topmargin}{-0.4in}
\setlength{\textheight}{8.5in}


%%%%%%%%%%%%%%%%%%%%%%%%%%%%%%%%%%%%%%%%%%%%%%%%%%%%%%%%%%%%%%%%%%%%%%%%%%%%%%%%%%%
\newcommand{\notes}[5]{
	\renewcommand{\thepage}{#1 - \arabic{page}}
	\noindent
	\begin{center}
	\framebox{
		\vbox{
		\hbox to 5.78in { { \bf Statistical Machine Learning}
		\hfill #2}
		\vspace{4mm}
		\hbox to 5.78in { {\Large \hfill #5 \hfill} }
		\vspace{2mm}
		\hbox to 5.78in { {\it #3 \hfill #4} }
		}
	}
	\end{center}
	\vspace*{4mm}
}

\newcommand{\ho}[5]{\notes{#1}{Probability}{Professor: Zhihua Zhang}{}{Lecture Notes #1: Probability}}
%%%%%%%%%%%%%%%%%%%%%%%%%%%%%%%%%%%%%%%%%%%%%%%%%%%%%%%%%%%%%%%%%%%%%%%%%%%%%%%%%%

%begins a LaTeX document
\begin{document}

\ho{1}{2011.02.21}{Moses Liskov}{Name}{Lecture title}

\section{Probability Theory Basics}
\subsection{Sample Space and Events}
\begin{definition}
The \textit{sample space} $\Omega$ is the set of possible outcomes of an experiment,
$\omega \in \Omega$ are called sample outcomes, realizations or elements.
The subsets of $\Omega$ are called \textit{events}.
\end{definition}

\begin{definition}
Given an event, $A\subset \Omega$, let $A^c = \{\omega \in \Omega, \omega \notin A\}$
denote the complement of $A$.
\end{definition}

\begin{definition}
A sequence of sets $A_1, A_2, \cdots$ is \emph{monotone increasing},
if $A_1 \subset A_2 \subset \cdots$, we define $\displaystyle\lim_{n\rightarrow \infty} A_n = \bigcup_{i = 1}^{\infty}A_i$.
\end{definition}

\begin{definition}
A sequence of sets $A_1, A_2, \cdots$ is \emph{monotone decreasing},
if $A_1 \supset A_2 \supset \cdots$, we define $\displaystyle\lim_{n\rightarrow \infty} A_n = \bigcap_{i = 1}^{\infty}A_i$.
\end{definition}

\begin{example}
Let $\Omega = \R$ and $A_i = [0, 1/i)$ for $i = 1, 2\cdots$,then \\
$\displaystyle \bigcup_{i = 1}^{\infty}A_i = [0, 1), \displaystyle \bigcap_{i = 1}^{\infty}A_i = \{0\}$.
If $A_i = (0, 1/i)$, then $\displaystyle \bigcup_{i = 1}^{\infty}A_i = (0, 1), \displaystyle \bigcap_{i = 1}^{\infty}A_i = \emptyset$.
\end{example}

\subsection{$\sigma$-field and Measures}
\begin{definition}
Let $\MA$ be a collection of subsets of a sample space $\Omega$. $\MA$ is called $\sigma$-field (or $\sigma$-algebra).
iff
\begin{enumerate}
\item The empty set $\emptyset \in \MA$.
\item If $A \in \MA$, $A^c \in \MA$.
\item If $A_i \in \MA$, $i \in \{1,2, ..., k\}$, then $\displaystyle\bigcup_{i=1}^k A_i \in \MA$.
\end{enumerate}
\end{definition}

\begin{definition}
A pair $(\Omega, \MA)$ is called a measurable space.
\end{definition}

\begin{example}
Let $A$ be a nonempty proper subset of $\Omega$, i.e. $A \neq \emptyset$, $A\neq \Omega$,
the smallest $\MA = \{\emptyset, \Omega, A, A^c\}$.
\end{example}

\begin{example}
$\Omega = \BR$. The smallest $\sigma$-field that contains all the finite open sets of $\BR$  is called Borel $\sigma$-field.
\end{example}

\begin{definition}
Let $(\Omega, \MA)$ be a measurable space. A set function $\nu$ defined on $\MA$ is called a measure iff
\begin{enumerate}
\item $0 \leq \nu(A) \leq \infty$ for any $A \in \MA$.
\item $\nu(\emptyset) = 0$.
\item If $A \in \MA$, and $A_i$ are disjoint, i.e. $A_i \cap A_j = \emptyset$ for $i \neq j$,
then $\nu(\bigcup_{i=1}^\infty A_i) = \sum_{i=1}^\infty \nu(A_i)$.
\end{enumerate}
\end{definition}

\begin{definition}
Tripe $(\Omega, \MA, \nu)$ is called a measure space.

If $\nu(\Omega) = 1$, then $\nu$ is called a probability measure and denote it by $P$.
$(\Omega, \MA, P)$ is called a probability space.
\end{definition}

\begin{example}
Let $\Omega$ be a sample space, $\MA$ is a collection of all subsets, and $\nu(A)$ is the number of elements in $A$.
\end{example}

\begin{lemma}
For any two events $A$ and $B$. $P(A \cup B) = P(A) + P(B) - P(A \cap B)$.
\end{lemma}

\begin{theorem}[Continuity of Probability]
If $A_n \rightarrow A$, then
\[ P(A_n) \rightarrow P(A) \text{ as } n\rightarrow \infty. \]
\end{theorem}
{\bf Proof: } We first consider the case where $A_n$ is monotone increasing.\\
Recall that $A_1\subset A_2\dots$ and let $A = \lim_{n\rightarrow \infty}A_n = \bigcup_{i = 1}^{\infty}A_i$.\\
Define $B_1 = A_1$, $B_2 = \{ \omega \in \Omega : \omega \in A_2, \omega \notin A_1\}$,
$B_3 = \{ \omega \in \Omega : \omega \in A_3, \omega \notin A_2\}$\dots.
Then for each $n$, we have $A_n = \bigcup_{i=1}^{n}A_i = \bigcup_{i=1}^{n}B_i$.\\
Thus, $\bigcup_{i=1}^{\infty}A_i = \bigcup_{i=1}^{\infty}B_i$. So that,
\[P(A_n) = \sum_{i=1}^n P(B_i)\]
Hence, we have,
\begin{align}\nonumber
\lim_{n\rightarrow\infty}P(A_n) & = \lim_{n\rightarrow\infty}\sum_{i=1}^n P(B_i) = \sum_{i=1}^{\infty}P(B_i)\\
& = P(\bigcup_{i=1}^{\infty}B_i) = P(A)\nonumber
\end{align}
For arbitrary sequence $\{A_i\}$, we can define $\{C_i\}$ to construct a monotone increasing sequence.
Specifically,
$C_1 = A_1 \cap A$,
$C_2 = (A_1 \cup A_2) \cap A$,
$C_3 = (A_1 \cup A_2 \cup A_3) \cap A$,\dots


\subsection{Independent Events}
\begin{definition}
Two events $A$ and $B$ are independent if $P(A\cap B) = P(A)P(B)$.
\end{definition}
We write $A \probin B$ to denote independence. For a set of events $\{A_i, i\in I\}$ $A$, it is
independent if $P(\bigcap_{i\in J}A_i) = \prod_{i\in J}P(A_i)$, for
every finite subset $J$ of $I$.

\subsection{Conditional Probability}
\begin{definition}
If $P(B) > 0$, the conditional probability of $A$ given $B$ is
\[P(A|B) = \frac{P(A\cap B)}{P(B)}.\]
\end{definition}

\begin{lemma}
If $A$ and $B$ are independent events, then $P(A|B) = P(A)$. Also, for any events $A$, $B$
\[P(AB) = P(A|B)P(B) = P(B|A)P(A).\]
\end{lemma}

\subsection{Bayes Theorem}
\begin{theorem}[The Law of Total Probability]
Let $A_1, A_2, \dots, A_k$ be partition of $\Omega$. Then
for any event $B$,
\[P(B) = \sum_{i=1}^k P(B|A_i)P(A_i)\]
\end{theorem}
{\bf Proof: } Define $C_j = B\cap A_j$ for $j = 1, \dots, k$.
Then we have $C_j \cap C_i = \emptyset$ and $B = \bigcup_{i = 1}^k C_i$.
Thus,\[P(B) = \sum P(C_j) = \sum P(B\cap A_j) = \sum P(B|A_j)P(A_j)\]

\begin{theorem}[Bayes Theorem]
Let $A_1, \dots, A_k$ be a partition of $\Omega$, such that $P(A_i) > 0$ for each $i$.
If $P(B) > 0$, then for each $i = 1, \dots, k$
\[P(A_i|B) = \frac{P(B|A_i)P(A_i)}{\sum_{i=1}^kP(B|A_j)P(A_j)}\]
\end{theorem}
{\bf Remarks:} We usually call those probabilities as
\begin{itemize}
\item $P(A_i)$ - prior probability of $A_i$
\item $P(A_i|B)$ - posterior probability of $A_i$
\item $P(B|A_i)$ - likelihood
\end{itemize}

\end{document}


