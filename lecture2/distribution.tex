\documentclass[11pt]{article}
\usepackage{graphicx} % more modern
%\usepackage{times}
\usepackage{helvet}
\usepackage{courier}
\usepackage{epsf}
\usepackage{amsmath,amssymb,amsfonts,verbatim}
\usepackage{subfigure}
\usepackage{amsfonts}
\usepackage{amsmath}
\usepackage{latexsym}
\usepackage{algpseudocode}
\usepackage{algorithm}
\usepackage{enumerate}
%\usepackage{algorithmic}
\usepackage{multirow}
\usepackage{xcolor}

\def\A{{\bf A}}
\def\a{{\bf a}}
\def\B{{\bf B}}
\def\b{{\bf b}}
\def\C{{\bf C}}
\def\c{{\bf c}}
\def\D{{\bf D}}
\def\d{{\bf d}}
\def\E{{\bf E}}
\def\e{{\bf e}}
\def\F{{\bf F}}
\def\f{{\bf f}}
\def\G{{\bf G}}
\def\g{{\bf g}}
\def\k{{\bf k}}
\def\K{{\bf K}}
\def\H{{\bf H}}
\def\I{{\bf I}}
\def\L{{\bf L}}
\def\M{{\bf M}}
\def\m{{\bf m}}
\def\n{{\bf n}}
\def\N{{\bf N}}
\def\BP{{\bf P}}
\def\R{{\bf R}}
\def\BS{{\bf S}}
\def\s{{\bf s}}
\def\t{{\bf t}}
\def\T{{\bf T}}
\def\U{{\bf U}}
\def\u{{\bf u}}
\def\V{{\bf V}}
\def\v{{\bf v}}
\def\W{{\bf W}}
\def\w{{\bf w}}
\def\X{{\bf X}}
\def\Y{{\bf Y}}
\def\Q{{\bf Q}}
\def\x{{\bf x}}
\def\y{{\bf y}}
\def\Z{{\bf Z}}
\def\z{{\bf z}}
\def\0{{\bf 0}}
\def\1{{\bf 1}}


\def\hx{\hat{\bf x}}
\def\tx{\tilde{\bf x}}
\def\ty{\tilde{\bf y}}
\def\tz{\tilde{\bf z}}
\def\hd{\hat{d}}
\def\HD{\hat{\bf D}}

\def\MA{{\mathcal A}}
\def\MF{{\mathcal F}}
\def\MR{{\mathcal R}}
\def\MG{{\mathcal G}}
\def\MI{{\mathcal I}}
\def\MN{{\mathcal N}}
\def\MO{{\mathcal O}}
\def\MT{{\mathcal T}}
\def\MX{{\mathcal X}}
\def\SW{{\mathcal {SW}}}
\def\MW{{\mathcal W}}
\def\MY{{\mathcal Y}}
\def\BR{{\mathbb R}}
\def\BP{{\mathbb P}}

\def\bet{\mbox{\boldmath$\beta$\unboldmath}}
\def\epsi{\mbox{\boldmath$\epsilon$}}

\def\etal{{\em et al.\/}\,}
\def\tr{\mathrm{tr}}
\def\rk{\mathrm{rk}}
\def\diag{\mathrm{diag}}
\def\dg{\mathrm{dg}}
\def\argmax{\mathop{\rm argmax}}
\def\argmin{\mathop{\rm argmin}}
\def\vecd{\mathrm{vec}}

\def\ph{\mbox{\boldmath$\phi$\unboldmath}}
\def\vp{\mbox{\boldmath$\varphi$\unboldmath}}
\def\pii{\mbox{\boldmath$\pi$\unboldmath}}
\def\Ph{\mbox{\boldmath$\Phi$\unboldmath}}
\def\pss{\mbox{\boldmath$\psi$\unboldmath}}
\def\Ps{\mbox{\boldmath$\Psi$\unboldmath}}
\def\muu{\mbox{\boldmath$\mu$\unboldmath}}
\def\Si{\mbox{\boldmath$\Sigma$\unboldmath}}
\def\lam{\mbox{\boldmath$\lambda$\unboldmath}}
\def\Lam{\mbox{\boldmath$\Lambda$\unboldmath}}
\def\Gam{\mbox{\boldmath$\Gamma$\unboldmath}}
\def\Oma{\mbox{\boldmath$\Omega$\unboldmath}}
\def\De{\mbox{\boldmath$\Delta$\unboldmath}}
\def\de{\mbox{\boldmath$\delta$\unboldmath}}
\def\Tha{\mbox{\boldmath$\Theta$\unboldmath}}
\def\tha{\mbox{\boldmath$\theta$\unboldmath}}

\newtheorem{theorem}{Theorem}[section]
\newtheorem{lemma}{Lemma}[section]
\newtheorem{definition}{Definition}[section]
\newtheorem{proposition}{Proposition}[section]
\newtheorem{corollary}{Corollary}[section]
\newtheorem{example}{Example}[section]


\def\probin{\mbox{\rotatebox[origin=c]{90}{$\vDash$}}}

\def\calA{{\cal A}}



%this is a comment

%use this as a template only... you may not need the subsections,
%or lists however they are placed in the document to show you how
%do it if needed.


%THINGS TO REMEMBER
%to compile a latex document - latex filename.tex
%to view the document        - xdvi filename.dvi
%to create a ps document     - dvips filename.dvi
%to create a pdf document    - dvipdf filename.dvi
%{\bf TEXT}                  - bold font TEXT
%{\it TEXT}                  - italic TEXT
%$ ... $                     - places ... in math mode on same line
%$$ ... $$                   - places ... in math mode on new line
%more info at www.cs.wm.edu/~mliskov/cs423_fall04/tex.html


\setlength{\oddsidemargin}{.25in}
\setlength{\evensidemargin}{.25in}
\setlength{\textwidth}{6in}
\setlength{\topmargin}{-0.4in}
\setlength{\textheight}{8.5in}


%%%%%%%%%%%%%%%%%%%%%%%%%%%%%%%%%%%%%%%%%%%%%%%%%%%%%%%%%%%%%%%%%%%%%%%%%%%%%%%%%%%
\newcommand{\notes}[5]{
	\renewcommand{\thepage}{#1 - \arabic{page}}
	\noindent
	\begin{center}
	\framebox{
		\vbox{
		\hbox to 5.78in { { \bf Statistical Machine Learning}
		\hfill #2}
		\vspace{4mm}
		\hbox to 5.78in { {\Large \hfill #5 \hfill} }
		\vspace{2mm}
		\hbox to 5.78in { {\it #3 \hfill #4} }
		}
	}
	\end{center}
	\vspace*{4mm}
}

\newcommand{\ho}[5]{\notes{#1}{Distribution}{Professor: Zhihua Zhang}{}{Lecture Notes #1: Distribution}}
%%%%%%%%%%%%%%%%%%%%%%%%%%%%%%%%%%%%%%%%%%%%%%%%%%%%%%%%%%%%%%%%%%%%%%%%%%%%%%%%%%

%begins a LaTeX document
\begin{document}

\ho{2}{2014.03.02}{Moses Liskov}{Name}{Lecture title}

\section{A}

\section{Discrete Distribution Examples}
\subsection{Uniform Discrete Distribution}
Random variable $\X \in \{x_1, x_2, ..., x_n\}$ has a uniform discrete distribution pmf $f$ if 
\[f(x) = \left\{\begin{array}{cc}
\frac{1}{n}  &  x = x_i, i = 1, 2, ..., n \\
0                &  \text{otherwise}
\end{array}
\right.\] 

\subsection{Point Mass Distribution}
Random variable $\X$ has a point mass distribution pmf $f$ if 
\[f(x) = \left\{\begin{array}{cc}
1 & x= a \\
0 & \text{otherwise}
\end{array}
\right.
\]

\subsection{Bernoulli Distribution}
Random variable $\X$ has a bernoulli distribution pmf $f$ if 
\[f(x) = \left\{\begin{array}{cc}
p & x= a \\
1-p & \text{otherwise}
\end{array}
\right.
\]
where $p \in [0,1]$.

\subsection{Poisson Distribution}
A discrete random variable $\X$ is said to have a Poisson distribution with parameter $\lambda > 0$, 
if, for $k = 0, 1, 2, ...$, the probability mass function of $\X$ is given by:
\[f(k; \lambda) = \Pr(X=k)= \frac{\lambda^k e^{-\lambda}}{k!}\].

{\bf Remark}: If $\X_1 \sim Poisson(\lambda_1)$, $\X_2 \sim Poisson(\lambda_2)$, 
then $\X_1 + \X_2 \sim Poisson(\lambda_1 + \lambda_2)$.

\subsection{Binomial Distribution}
A discrete random variable $\X$ is said to have a binomial distribution with parameter $n$ and $p$, we write $\X \sim Binomial(n,p))$. The probability mass function is given by:
\[f(k;n,p) = \Pr(\X = k) = {n\choose k}p^k(1-p)^{n-k}\]
for $k=0,1,2,...,n$, where ${n \choose k} = \frac{n!}{k!(n-k)!}$ is the binomial coefficient. It can be interpreted that the probability of exact $k$ successes after $n$ trials.

[ need to add more ... ]

\subsection{Negative Binomial Distribution}
Suppose there is a sequence of independent Bernoulli trials, each trial having two potential outcomes called ``success" and ``failure". In each trial the probability of success is $p$ and of failure is $1 − p$. We are observing this sequence until a predefined number $r$ of failures has occurred. Then the random number of successes we have seen, $\X$, will have the negative binomial (or Pascal) distribution:
\[\X \sim NB(r, p).\]

The probability mass function of the negative binomial distribution is:
\[f(k; r,p) = Pr(\X = k) = {k+r-1 \choose k} p^k (1-p)^r\]
for $k = 0, 1, 2, ...$

\subsection{Geometric Distribution}


\section{Continuous Distribution Examples}
\subsection{Continuous Uniform Distribution}
A continuous random variable $\X$ is said to have a uniform distribution in $[a,b]$, 
if the probability density function is given by:
\[ f(x)=\begin{cases}
  \frac{1}{b - a} &  a \leq x \leq b \\
  0 & \mathrm{otherwise}
  \end{cases} \]
  
\subsection{Normal(Gaussian) Distribution}
A continuous random variable $\X$ is said to have a Gaussian distribution with parameter $\mu$ and $\sigma$, if the probability density function of $\X$ is given by:
\[f(x; \mu, \sigma) = \frac{1}{\sqrt{2\pi}\sigma} e^{-\frac{(x-\mu)^2}{2\sigma^2}}\]
The cumulative distribution function of Gaussian random variable $\X$ with parameter $\mu = 0$ and $\sigma = 1$ is:
\[\Phi(z) = Pr(\X < z) = \int_{-\infty}^z \frac{1}{\sqrt{2\pi}} e^{-\frac{t^2}{2}} dt\]

\subsection{Disac Distribution}

\subsection{Exponential Power Distribution}

\subsection{Generalized Inverse Gaussian Distribution}
A continuous random variable $\X$ is said to have generalized inverse Gaussian distribution with parameters $\alpha$, $\beta$, $r$, if the probability density function of $\X$ is given by:
\[f(x) = \frac{(\alpha / \beta)^{r/2}}{2 K_r(\sqrt{\alpha\beta})} x^{r-1} e^{-(\alpha x + b/x)/2}, x > 0\]
where $K_r$ is a modified Bessel function of third kind with index $r$, $\alpha > 0$, $\beta > 0$.

\subsection{Chi-Squared Distribution}
A continuous random variable $\X$ is said to have chi-squared distribution, if the probability density function of $\X$ is given by:
\[f(x) = \frac{1}{\Gamma(\frac{p}{2})2^{\frac{p}{2}}} x^{\frac{p}{2} - 1} e^{-\frac{x}{2}}\]

\subsection{Beta Distribution}
A continuous random variable $\X$ is said to have beta distribution, if the probability density function of $\X$ is given by:
\[f(x) = \frac{\Gamma(\alpha + \beta)}{\Gamma(\alpha)\Gamma(\beta)} x^{\alpha - 1} (1-x)^{\beta - 1}\]

\end{document}




