\documentclass[11pt]{article}
\usepackage{graphicx} % more modern
%\usepackage{times}
\usepackage{helvet}
\usepackage{courier}
\usepackage{epsf}
\usepackage{amsmath,amssymb,amsfonts,verbatim}
\usepackage{subfigure}
\usepackage{amsfonts}
\usepackage{amsmath}
\usepackage{amsthm}
\usepackage{latexsym}
\usepackage{algpseudocode}
\usepackage{algorithm}
\usepackage{enumerate}
%\usepackage{algorithmic}
\usepackage{multirow}
\usepackage{xcolor}
\usepackage{fancyref}


\def\A{{\bf A}}
\def\a{{\bf a}}
\def\B{{\bf B}}
\def\b{{\bf b}}
\def\C{{\bf C}}
\def\c{{\bf c}}
\def\D{{\bf D}}
\def\d{{\bf d}}
\def\E{{\bf E}}
\def\e{{\bf e}}
\def\F{{\bf F}}
\def\f{{\bf f}}
\def\G{{\bf G}}
\def\g{{\bf g}}
\def\k{{\bf k}}
\def\K{{\bf K}}
\def\H{{\bf H}}
\def\I{{\bf I}}
\def\L{{\bf L}}
\def\M{{\bf M}}
\def\m{{\bf m}}
\def\n{{\bf n}}
\def\N{{\bf N}}
\def\BP{{\bf P}}
\def\R{{\bf R}}
\def\BS{{\bf S}}
\def\s{{\bf s}}
\def\t{{\bf t}}
\def\T{{\bf T}}
\def\U{{\bf U}}
\def\u{{\bf u}}
\def\V{{\bf V}}
\def\v{{\bf v}}
\def\W{{\bf W}}
\def\w{{\bf w}}
\def\X{{\bf X}}
\def\Y{{\bf Y}}
\def\Q{{\bf Q}}
\def\x{{\bf x}}
\def\y{{\bf y}}
\def\Z{{\bf Z}}
\def\z{{\bf z}}
\def\0{{\bf 0}}
\def\1{{\bf 1}}


\def\hx{\hat{\bf x}}
\def\tx{\tilde{\bf x}}
\def\ty{\tilde{\bf y}}
\def\tz{\tilde{\bf z}}
\def\hd{\hat{d}}
\def\HD{\hat{\bf D}}
\def\px {\partial{x}}
\def\py{\partial{y}}

\def\MA{{\mathcal A}}
\def\ML{{\mathcal L}}
\def\MF{{\mathcal F}}
\def\MR{{\mathcal R}}
\def\MG{{\mathcal G}}
\def\MI{{\mathcal I}}
\def\MN{{\mathcal N}}
\def\MO{{\mathcal O}}
\def\MT{{\mathcal T}}
\def\MX{{\mathcal X}}
\def\SW{{\mathcal {SW}}}
\def\MW{{\mathcal W}}
\def\MY{{\mathcal Y}}
\def\BR{{\mathbb R}}
\def\BP{{\mathbb P}}
\def\BE{{\mathbb E}}
\def\BN{{\mathbb N}}

\def\bet{\mbox{\boldmath$\beta$\unboldmath}}
\def\epsi{\mbox{\boldmath$\epsilon$}}

\def\etal{{\em et al.\/}\,}
\def\tr{\mathrm{tr}}
\def\rk{\mathrm{rk}}
\def\diag{\mathrm{diag}}
\def\dg{\mathrm{dg}}
\def\argmax{\mathop{\rm argmax}}
\def\argmin{\mathop{\rm argmin}}
\def\vecd{\mathrm{vec}}

\def\ph{\mbox{\boldmath$\phi$\unboldmath}}
\def\vp{\mbox{\boldmath$\varphi$\unboldmath}}
\def\pii{\mbox{\boldmath$\pi$\unboldmath}}
\def\Ph{\mbox{\boldmath$\Phi$\unboldmath}}
\def\pss{\mbox{\boldmath$\psi$\unboldmath}}
\def\Ps{\mbox{\boldmath$\Psi$\unboldmath}}
\def\muu{\mbox{\boldmath$\mu$\unboldmath}}
\def\Si{\mbox{\boldmath$\Sigma$\unboldmath}}
\def\lam{\mbox{\boldmath$\lambda$\unboldmath}}
\def\Lam{\mbox{\boldmath$\Lambda$\unboldmath}}
\def\Gam{\mbox{\boldmath$\Gamma$\unboldmath}}
\def\Oma{\mbox{\boldmath$\Omega$\unboldmath}}
\def\De{\mbox{\boldmath$\Delta$\unboldmath}}
\def\de{\mbox{\boldmath$\delta$\unboldmath}}
\def\Tha{\mbox{\boldmath$\Theta$\unboldmath}}
\def\tha{\mbox{\boldmath$\theta$\unboldmath}}
\def\aph{\mbox{\boldmath$\alpha$\unboldmath}}
\def\bt{\mbox{\boldmath$\beta$\unboldmath}}

\newtheorem{theorem}{Theorem}[section]
\newtheorem{lemma}{Lemma}[section]
\newtheorem{definition}{Definition}[section]
\newtheorem{proposition}{Proposition}[section]
\newtheorem{corollary}{Corollary}[section]
\newtheorem{example}{Example}[section]


\def\probin{\mbox{\rotatebox[origin=c]{90}{$\vDash$}}}

\def\calA{{\cal A}}



%this is a comment

%use this as a template only... you may not need the subsections,
%or lists however they are placed in the document to show you how
%do it if needed.


%THINGS TO REMEMBER
%to compile a latex document - latex filename.tex
%to view the document        - xdvi filename.dvi
%to create a ps document     - dvips filename.dvi
%to create a pdf document    - dvipdf filename.dvi
%{\bf TEXT}                  - bold font TEXT
%{\it TEXT}                  - italic TEXT
%$ ... $                     - places ... in math mode on same line
%$$ ... $$                   - places ... in math mode on new line
%more info at www.cs.wm.edu/~mliskov/cs423_fall04/tex.html


\setlength{\oddsidemargin}{.25in}
\setlength{\evensidemargin}{.25in}
\setlength{\textwidth}{6in}
\setlength{\topmargin}{-0.4in}
\setlength{\textheight}{8.5in}


%%%%%%%%%%%%%%%%%%%%%%%%%%%%%%%%%%%%%%%%%%%%%%%%%%%%%%%%%%%%%%%%%%%%%%%%%%%%%%%%%%%
\newcommand{\notes}[5]{
	\renewcommand{\thepage}{#1 - \arabic{page}}
	\noindent
	\begin{center}
	\framebox{
		\vbox{
		\hbox to 5.78in { { \bf Statistical Machine Learning}
		\hfill #2}
		\vspace{4mm}
		\hbox to 5.78in { {\Large \hfill #5 \hfill} }
		\vspace{2mm}
		\hbox to 5.78in { {\it #3 \hfill #4} }
		}
	}
	\end{center}
	\vspace*{4mm}
}

\newcommand{\ho}[5]{\notes{#1}{Distributions}{Professor: Zhihua Zhang}{}{Lecture Notes #1: Jacobian and Wedge}}
%%%%%%%%%%%%%%%%%%%%%%%%%%%%%%%%%%%%%%%%%%%%%%%%%%%%%%%%%%%%%%%%%%%%%%%%%%%%%%%%%%

%begins a LaTeX document
\setcounter{section}{5}
\setcounter{subsection}{0}
\begin{document}
\ho{5}{2014.03.31}{Moses Liskov}{Name}{Lecture title}
\subsection{More About Mixture Distribution}
\begin{theorem}
	Let $X$ be an $m\times 1$ random vector having a density function $f_{\X}(\x)$,  which is positive on a set $\MX\subset\BR^m$. Suppose the transform $\y=\y(\x)=(y_1(\x),y_2(\x),\dots,y_m(\x))^T$ is 1-1 for some $\MY$, where $\MY$ denotes the image of $\MX$ under $\y$, s.t. the inverse transformation $\x=\x(\y)$ exists for $\y\in\MY$. Assuming that the partial derivatives $\frac{\px_i}{\py_j}$'s, $(i,j=1,2,\dots,m)$ exist and continuous on $\MY$, it is well known that the density function of random vector $\Y=\y(\X)$ is
	\begin{equation}
		f_{\Y}(\y) = f_{\X}(\x(\y))|J(\x\rightarrow \y)|,\ \y\in\MY
	\end{equation}
	where $J(\x\rightarrow \y)$ is the determinant of the Jacobian of the transformation, i.e.
	\begin{equation}
		J(\x\rightarrow \y) = \det 
		\begin{pmatrix} 
			\frac{\px_1}{\py_1}	& \frac{\px_1}{\py_2}	& \dots 	& \frac{\px_1}{\py_m}\\
			\frac{\px_2}{\py_1}	& \frac{\px_2}{\py_2}	& \dots 	& \frac{\px_2}{\py_m}\\
			\vdots 			& \vdots 			& \ddots 	& \vdots \\
			\frac{\px_m}{\py_1}	& \frac{\px_m}{\py_2}	& \dots 	& \frac{\px_m}{\py_m}\\
		\end{pmatrix}
	\end{equation}
\end{theorem}

Then we define the exterior product, or wedge product, which is a useful tool to calculate the determinant.
\begin{definition} [Wedge Product]
	The exterior product or wedge product of $dx$ and $dy$, denoted as $dx\land dy$, has the property that $dx \land dy = -dy \land dx$. 
\end{definition}

Therefore, it's easy to prove that when $x=y$, $dx\land dx=0$.

\begin{theorem}
	If $d\y=(dy_1,dy_2,\dots, dy_m)^T$ is an $m\times 1$ vector of differentials and if $d\x=(dx_1,dx_2,\dots, dx_m)^T = \B dy$, where $\B$ is an $m\times m$ nonsingular matrix, then
	\begin{equation}	\label{eq: detThm}
		\bigwedge_{i=1}^m dx_i=\det(\B) \bigwedge_{i=1}^m dy_i
	\end{equation}
	
\end{theorem}
\begin{proof}
	We'll prove it by induction.

	 $m=2$:
	\begin {equation}
		\begin{pmatrix}dx_1\\ dx_2 \end{pmatrix} 
		=
		 \begin{pmatrix}
			B_{11} & B_{12}\\
			B_{21} & B_{22}\\
		\end{pmatrix}
		\begin{pmatrix}dy_1\\ dy_2\\ \end{pmatrix}
		=  \begin{pmatrix}
			B_{11}dy_1+ B_{12}dy_2\\
			B_{21}dy_1+ B_{22}dy_2\\
		     \end{pmatrix} 
	\end{equation}
	Since $dy_i\wedge dy_i=0$ and $dy_2\wedge dy_1=-dy_1\wedge dy_2$, we have
	\begin{align}
		dx_1\wedge dx_2 &= (B_{11}dy_1+ B_{12}dy_2)\wedge (B_{21}dy_1+ B_{22}dy_2) \\
		&= (B_{11}B_{22}-B_{12}B_{21})dy_1\wedge dy_2 = \det(\B) dy_1\wedge dy_2
	\end{align}
	Suppose eq.~\ref{eq: detThm} holds for $m-1$. Now consider the case of $m$.
	Let $\B=	\begin{pmatrix}
				\A_{(m-1)\times (m-1)} & \b\\
				\a^T 				 & B_{mm}\\
			\end{pmatrix}$
	and $\Q = 	\begin{pmatrix} 
				\I_{(m-1)\times (m-1)} & -\b B_{mm}^{-1}\\
				0				& 1\\
			\end{pmatrix}$.
	\begin{alignat*}{3}
		 &		& d\x 		&= 	\begin{pmatrix}
									\A_{(m-1)\times (m-1)} & \b\\
									\a^T 				 & B_{mm}\\
								\end{pmatrix} d\y \\
		&\implies	& \Q d\x 		&= \Q \B d\y			\\
		&\implies	& \begin{pmatrix} 
					d\x_{1:(m-1)} - B_{mm}^{-1}\b dx_m\\
					dx_m\\
				   \end{pmatrix}
							&= \begin{pmatrix}
								(\A-B_{mm}^{-1}\b\a^T)d\y_{1:(m-1)} \\
								\a^T d\y_{1:(m-1)}+B_{mm}dy_m\\
							     \end{pmatrix}
	\end{alignat*}
	where $d\x_{1:(m-1)} = (dx_1,\dots,dx_{m-1})$ and  $d\y_{1:(m-1)}$ is defined similarly.
	Calculating the wedge for both sides, we have
	\begin{align*}
		\bigwedge_{i=1}^m dx_i &= \bigwedge((\A-B_{mm}^{-1}\b\a^T)d\y_{1:(m-1)}) \bigwedge (B_{mm} dy_m)\\
		&= B_{mm} \det(\A-B_{mm}^{-1}\b\a^T) \bigwedge_{i=1}^m dy_i\\
		&= \det \begin{pmatrix}
				\A-B_{mm}^{-1}\b\a^T & \bf{0} \\
				\a^T  			& B_{mm}\\
			\end{pmatrix} \bigwedge_{i=1}^m dy_i\\
		&= \det(\Q\B) \bigwedge_{i=1}^m dy_i = \det(\Q)\det(\B)\bigwedge_{i=1}^m dy_i = \det(\B)\bigwedge_{i=1}^m dy_i\\
	\end{align*}
\end{proof}


\begin{example}
	Convert rectangular coordinates $x_1,x_2,\dots, x_m$ to polar coordinates $r,\theta_1,\theta_2,\dots, \theta_{m-1}$, where
	\begin{align}
		x_1 &= r sin\theta_1 sin\theta_2\cdots sin\theta_{m-2} sin\theta_{m-1}\\
		x_2 &= r sin\theta_1 sin\theta_2\cdots sin\theta{m-2} cos\theta_{m-1}\\	
		\vdots\\
		x_{m-1} &= r sin\theta_1 cos\theta_2\\
		x_m &= r cos\theta_1\\
		&(r>0, 0<\theta_i \leq \pi (i=1,2,\dots,m-2), 0< \theta_{m-1}\leq 2\pi).
	\end{align}
	Then 
	\begin{equation} \label{eq:recToPolar}
		J(\x\rightarrow r,\theta_1,\dots,\theta_{m-1}) = r^{m-1} sin^{m-2}\theta_1 sin^{m-3}\theta_2\cdots sin\theta_{m-2}.
	\end{equation}
\end{example}
\begin{proof}
	\begin{alignat*}{3}
		&x_1^2 &&= r^2 sin^2\theta_1 sin^2\theta_2 \cdots sin^2\theta_{m-2} sin^2\theta_{m-1}\\
		&x_1^2+x_2^2 &&= r^2 sin^2\theta_1 sin^2\theta_2 \cdots sin^2\theta_{m-2}\\
		&\vdots &&\quad \vdots\\
		&x_1^2+\cdots +x_{m-1}^2 &&= r^2 sin^2\theta_1\\
		&x_1^2+\cdots+x_{m-1}^2+x_m^2 &&= r^2\\
	\implies & 2x_1dx_1 &&= 2r^2 sin^2\theta_1 \cdots sin^2\theta_{m-2} sin\theta_{m-1} cos\theta_{m-1} d\theta_{m-1}+ \text{terms\ of\ } dr,d\theta_1,\dots, d\theta_{m-2}\\
		& 2x_1dx_1+2x_2dx_2 &&= 2r^2 sin^2\theta_1 \cdots sin^2\theta_{m-3} sin\theta_{m-2} cos\theta_{m-2} d\theta_{m-2}+ \text{terms\ of\ } dr,d\theta_1,\dots, d\theta_{m-3}\\
		&\vdots &&\quad \vdots\\
		& 2x_1dx_1+\cdots+2x_mdx_m  &&= 2rdr
	\end{alignat*}
	Take wedge product for LHS and RHS simultaneously. we have
	\begin{align}
		2^m\prod_{i=1}^m x_i \bigwedge_{i=1}^m dx_i &=  2^m r^{2m-1} sin^{2m-3}\theta_1\cdots sin\theta_{m-1} \prod_{i=1}^{m-1}cos\theta_i\ dr\bigwedge(\bigwedge_{i=1}^{m-1} d\theta_i)
	\end{align}
	where
	\begin{equation}
		\prod_{i=1}^m x_i = r^m sin^{m-1}\theta_1\cdots sin\theta_{m-1}\prod_{i=1}^{m-1} cos\theta_i.
	\end{equation}
	Then we can prove eq.~\ref{eq:recToPolar} easily.
\end{proof}

\begin{definition}
	For any matrix $\X = (x_{ij})_{n\times m}$, $d\X\overset{def}{=} (dx_{ij})$,
	$d(\X\Y) = \X d\Y+d\X\Y$,
	where the symbol $(d\X) = \bigwedge_{i=1}^n\bigwedge_{j=1}^m dx_{ij}$.

	If $\X$ is a symmetric $m\times m$ matrix, the symbol $(d\X) = \bigwedge_{1\leq i\leq j\leq m} dx_{ij}$.
\end{definition}

\begin{theorem} \label{thm:asym}
	Let $\X$ and $\Y$ be two $n\times m$ matrix, and $\X = \B\Y\C$ where $B_{n\times n}$ and $C_{m\times m}$ are nonsingular. Then we have
	\begin{equation}
		(d\X) = (\det\B)^m(\det\C)^n(d\Y), \text{\ i.e.}\\
		J(\X\rightarrow \Y) =  (\det\B)^m(\det\C)^n\\
	\end{equation} 
\end{theorem}
\begin{proof}
	\begin{align}
		vec(\X) 	&= vec(\B\Y\C)\\
				&= (\C^T \otimes \B) vec(\Y),
	\end{align}
	where $\A\otimes \B = (a_{ij}\B)$ is the Kronecker product of $\A$ and $\B$ and $vec(\A)$ is the result of concatenating the columns of $\A$.
	\begin{align}
		(d\X) 	&= \det(\C^T \otimes \B) (d\Y)\\
				&= (\det\C)^n(\det\B)^m (d\Y)\\
	\end{align}
\end{proof}

Consider the symmetric case, we have the following.

\begin{theorem} \label{thm:sym}
	If $\X=\B\Y\B^T$ where $\X$ and $\Y$ are $m\times m$ symmetric matrices and $\B$ is a nonsingular matrix, then
	\begin{equation}
		(d\X) = (\det\B)^{m+1}(d\Y),
		J(\X\rightarrow\Y) = (\det\B)^{m+1}
	\end{equation}
\end{theorem}
\begin{proof}
	\begin{equation}
		d\X = \B d\Y\B^T \Rightarrow (d\X) = (\B d\Y\B^T) = \rho(\B)(d\Y)
	\end{equation}
	where $\rho(\B)$ is a polynomial of elements of $\B$.
	\begin{lemma}
		if $\rho$ is a polynomial and has the property that $\rho(\X_1\X_2) = \rho(\X_1)\rho(\X_2)$, then $\rho(\X)=(\det \X)^k$ for some $k$.
	\end{lemma}
	It's easy to show $\rho(\B_1\B_2)=\rho(\B_1)\rho(\B_2)$:
	\begin{equation}
		(\B_1\B_2d\Y\B_2^T\B_1^T) = (\B_1(\B_2d\Y\B_2^T)\B_1^T) = \rho(\B_1)(\B_2d\Y\B_2^T) = \rho(\B_1)\rho(\B_2) (d\Y)
	\end{equation}
	Because of this property, we have $\rho(\B) = (\det\B)^k $ for some $k$.
	To determine $k$, we consider a simple case where $\B = \diag(b,1,\dots,1)$ and let $\Y = (y_{ij})$. Then
	$$ \B\Y\B^T =
		\begin{pmatrix}
			b^2y_{11}	& b y_{12}	& \cdots 	& b y_{1m}\\
			b y_{21}	& y_{22}	& \cdots 	& y_{2m}\\
			\vdots 	& \vdots 	& \ddots 	& \vdots \\
			b y_{m1} 	& y_{m2} 	& \cdots 	& y_{mm}\\
		\end{pmatrix}.
	$$
	So considering the upper triangular part, we have
	$$(d\X) = (\B d\Y\B^T) = b^{m+1}(d\Y) = (\det\B)^{m+1}(d\Y).s$$
	Therefore, $k = m+1$.
\end{proof}

\begin{theorem}
	If $\X_{m\times m} = \Y^{-1}$, then 
		$$(d\X) = (\det\Y)^{-2m}(d\Y).$$
	Further, if $\Y$ is symmetric, then 
		$$(d\X) = (\det\Y)^{-(m+1)}(d\Y).$$
\end{theorem}
\begin{proof}
	Since 
		$$\X = \Y^{-1}\Y\Y^{-1},$$
	If $\Y$ is asymmetric, by Theorem.~\ref{thm:asym}, we have
		$$(d\X) = (\det\Y^{-1})^{2m}(d\Y) = (\det\Y)^{-2m}(d\Y).$$
	If $\Y$ is symmetric, by Theorem.~\ref{thm:sym}, we have
		$$(d\X) = (\det\Y^{-1})^{m+1}(d\Y) = (\det\Y)^{-(m+1)}(d\Y).$$
\end{proof}

\begin{theorem}
	If $\A$ is an $m\times m$ symmetric positive definite matrix, by Cholesky decomposition, $\A=\T^T\T$, where $\T =\left( \begin{smallmatrix} 
						t_{11}	& t_{12}	& \cdots 	& t_{1m}\\	
						\quad	& t_{22} 	& \cdots 	& t_{2m} \\
						\quad	& *		& \ddots 	& \vdots \\
						\quad 	&\quad 	& \quad	& t_{mm}\\
					  \end{smallmatrix}\right)$ is an upper triangular matrix with positive diagonal elements. Then we have 
	\begin{equation}
		J(\A\rightarrow \T) = 2^m \prod_{i=1}^m t_{ii}^{m+1-i}.
	\end{equation}
\end{theorem}
\begin{proof}
	Let $\A=(a_{ij})_{m\times m}$. 
	$$ 	
		\begin{pmatrix}
			a_{11}	& a_{12}	& \cdots 	& a_{1m}\\
			a_{21}	& a_{22}	& \cdots 	& a_{2m}\\
			\vdots 	& \vdots 	& \ddots 	& \vdots \\
			a_{m1} 	& a_{m2} 	& \cdots 	& a_{mm}\\
		\end{pmatrix}
		= 
		\begin{pmatrix} 
			t_{11}	&\quad	&\quad 	& \quad \\	
			t_{12}	& t_{22} 	& * 		& \quad \\
			\vdots	& \vdots	& \ddots 	& \quad \\
			t_{1m} 	& t_{2m}	& \cdots	& t_{mm}\\
		\end{pmatrix}
		\begin{pmatrix} 
			t_{11}	& t_{12}	& \cdots 	& t_{1m}\\	
			\quad	& t_{22} 	& \cdots 	& t_{2m} \\
			\quad	& *		& \ddots 	& \vdots \\
			\quad 	&\quad 	& \quad	& t_{mm}\\
		\end{pmatrix}
	$$
	So
	\begin{alignat*}{7}
		a_{11} &= t_{11}^2				&\implies	&& da_{11} &= 2t_{11}dt_{11}\\
		a_{12} &= t_{11}t_{12}			&\implies	&& da_{12} &= dt_{11}t_{12}+t_{11}dt_{12}\\
		\vdots  & \quad\vdots 				&\quad	&& \vdots    & \quad\vdots\\
		a_{ii}	   &= t_{1i}^2+\cdots+t_{ii}^2	&\implies	&& da_{ii}	  &= \text{terms\ of\ }dt_{1i},\dots,dt_{(i-1),i}+2t_{ii}dt_{ii}\\
		(i<j)\ a_{ij} &= t_{1i}t_{1j}+\cdots+t_{ii}t_{ij}	&\implies 	&& da_{ij} &= \text{terms\ of\ }dt_{1j},\dots,dt_{(i-1),j},dt_{1i},\dots,dt_{(i-1),i} +dt_{ii}t_{ij}+t_{ii}dt_{ij}\\
		\vdots  & \quad\vdots 				&\quad	&& \vdots    & \quad\vdots\\
		a_{mm}	   &= t_{1m}^2+\cdots+t_{mm}^2	&\implies	&& da_{mm}	  &= \text{terms\ of\ }dt_{1m},\dots,dt_{(m-1),m}+2t_{mm}dt_{mm}
	\end{alignat*}
	When taking wedge of both sides, we have
	\begin{equation*}
		(d\A) = 2^m \prod_{i=1}^m t_{ii}^{m+1-i} (d\T)
	\end{equation*}
\end{proof}


{\bf Homework}
\begin{enumerate}
	\item Compute the Laplace transforms of Gamma, Negative Binomial, Poisson distributions.
	\item Consider that
		\begin{alignat*}{5}
			w_1		&= w\alpha					&, && w_2 &=w(1-\alpha),\\
			u_1		&= u-\beta\sigma\sqrt{\frac{w_2}{w_1}}	&, && u_2 &=  u+\beta\sigma\sqrt{\frac{w_1}{w_2}}\\
			\sigma_1^2 &= r(1-\beta^2)\sigma^2w/w_1	&, && \sigma_2^2 &= (1-r)(1-\beta^2)\sigma^2w/w_2,
		\end{alignat*}
		where $\alpha,\beta,r\in(0,1)$.
		Compute the Jacobian from $(w_1,w_2,u_1,u_2,\sigma_1^2,\sigma_2^2)$ to $(w,u,\sigma^2,\alpha,\beta,r)$
\end{enumerate}

\subsection{Random variables and their properties}

\begin{definition}
	Let random vector $\X=(X_1,\dots,X_m)^T$. Then mean of $\X$ is 
		$$\mu=(\mu_1,\dots,\mu_m)^T=(\BE[X_1],\dots,\BE[X_m])^T.$$
	The covariance matrix $\Si$ is
		$$\Si = Cov(\X) = 
			\begin{pmatrix}
				Var(X_1)		& Cov(X_1,X_2)	& \cdots 	& Cov(X_1,X_m)\\
				Cov(X_1,X_2)	& Var(X_2)	 	& \cdots 	& Cov(X_2,X_m)\\	
				\vdots 		& \vdots 		& \ddots 	& \vdots\\
				Cov(X_m,X_1) 	& Cov(X_m,X_2)& \cdots 	& Var(X_m,X_m)
			\end{pmatrix}.
		$$
\end{definition}

\begin{lemma}
	If $\a$ is a vector and $\X$ is a random vector with mean $\muu$ and covariance matrix $\Si$, then
		$$\BE[\a^T\X]=\a^T\muu \quad \text{and} \quad Var(\a^T\X)=\a^T\Si\a$$.
	If $\A$ is a matrix then
		$$\BE[\A\X] = \A\muu  \quad \text{and} \quad Cov(\A\X) = \A\Si\A^T$$.
\end{lemma}

\subsubsection{Examples}
\begin{definition} [The Multinomial Distribution]
	A discrete random vector $\X = (X_1,\dots,X_k)$ has Multinomial distribution of dimension $k$ with parameters $\tha=(\theta_1,\dots,\theta_k)$ and $n$ ($0\leq \theta_i \leq 1, \sum_{i=1}^{k}\theta_i \leq1, n=1,2,\dots$).
	If it's p.d.f. is 
		\begin{equation}
			Mul(\x|\tha,n) = \frac{n!}{\prod_{i=1}^k x_i!(n-\sum_{i=1}^k x_i)!}\prod_{i=1}^k \theta_i^{x_i} (1-\sum_{t=1}^k\theta_t)^{n-\sum_{j=1}^kx_j}.
		\end{equation}
\end{definition}
The mean vector and covariance matrix
	\begin{align}
		\BE[X_i] &= n\theta_i\\
		Var(X_i) &= n\theta_i(1-\theta_i)\\
		Cov(X_i,X_j) &= -n\theta_i\theta_j.
	\end{align}

\begin{theorem}
	The marginal distribution of $\X^{(m)}=(X_1,\dots,X_m)^T, (m<k)$ is the multinomial distribution
		$$M_m(\x^{(m)}|(\theta_1,\dots,\theta_m),n).$$
	The conditional distribution of $\X^{(m)}$ given the remaining $X_i$'s is also Multinomial 
		$$f(x^{(m)}|x_{m+1},\cdots,x_k)\sim M_{m-1}(\x^{(m)}|(\theta'_1,\dots, \theta'_m),n-s),$$
	where $\theta'_i = \frac{\theta_i}{\sum_{j=1}^{m}\theta_j}, (1\leq i\leq m)$ and $s=\sum_{i=m+1}^{k}x_i$.
\end{theorem}

Its corresponding prior is
\begin{definition} [Dirichlet Distribution]
	A continuous random vector $\X=(X_1,\dots,X_k)$ has a Dirichlet distribution of dimension $k$ with parameter $\aph=(\alpha_1,\dots, \alpha_{k+1})$, $(\alpha_i>0, i=1,\dots,k+1)$
	if its p.d.f. is 
		\begin{equation}
			Dir(\x|\aph) = \frac{\Gamma(\sum_{i=1}^{k+1}\alpha_i)}{\prod_{i=1}^{k+1}\Gamma(\alpha_i)}x_1^{\alpha_1-1}\cdots x_k^{\alpha_k-1}(1-\sum_{i=1}^k x_i)^{\alpha_{k+1}-1}
		\end{equation}.
\end{definition}
The mean vector and covariance matrix
	\begin{align}
		\BE[X_i] &= \alpha_i/\sum_{j=1}^{k+1}\alpha_j\\
		Var(X_i) &= \frac{\BE[X_i](1-\BE[X_i])}{1+\sum_{j=1}^{k+1}\alpha_j}\\
		Cov(X_i,X_j) &= \frac{\BE[X_i]\BE[X_j]}{1+\sum_{t=1}^{k+1}\alpha_t}.
	\end{align}

\begin{theorem}
	The marginal distribution of $\X^{(m)}=(X_1,\dots,X_m)^T, (m<k)$ is the Dirichlet distribution
		$$Dir(\x^{(m)}|(\alpha_1,\dots,\alpha_m),\sum_{i=m+1}^{k+1}\alpha_i).$$
	The conditional distribution of $\X^{(m)}$ given the $X_{m+1},\dots,X_k$ of $Y_i = \frac{X_i}{1-\sum_{j=m+1}^k X_j}$ is also Dirichlet 
		$$Dir(\y^{(m)}|\alpha_1,\dots, \alpha_m,\alpha_{k+1}).$$
\end{theorem}

\begin{theorem}
	A random vector $\X=(X_1,\dots,X_k) \sim Dir(\x|\aph=(\alpha_1,\dots,\alpha_{k+1}))$. If $\Z=(Z_1,\dots,Z_t)$ satisfies 
	\begin{align}
		Z_1&=X_1+\cdots+X_{i_1}\\
		Z_2&=X_{i_1+1}+\cdots +X_{i_2}\\
		\vdots\\
		Z_t&=X_{i_{t-1}+1}+\cdots +X_k,
	\end{align}
	then 
		$$\Z \sim Dir(\z|\bt),$$
	where $\bt=(\beta_1,\dots,\beta_{t+1})$ and
		\begin{align*}
			\beta_1 &= \alpha_1+\cdots+\alpha_{i_1}\\
			\beta_2 &= \alpha_{i_1+1}+\cdots+\alpha_{i_2}\\
			\vdots\\
			\beta_t&= \alpha_{i_{t-1}+1}+\cdots +\alpha_k\\
			\beta_{t+1} &= \alpha_{k+1}.
		\end{align*}
\end{theorem}

{\bf Homework}
\begin{enumerate}
\item
Show the conditional distribution of multinomial distribution in Theorem 5.7.
\item
\begin{align*}
	\BP(\X|\tha,n)		&\sim Multinomial\ Distribution,\\
	\BP(\tha|\aph) 	&\sim Dirichlet\ Distribution.
\end{align*}
Compute $\BP(\theta|\X)$.
\end{enumerate}


\end{document}




